\documentclass[12pt,a4paper,bibliography=totocnumbered]{scrartcl}
\usepackage[ngerman]{babel}

\usepackage{graphicx}
\usepackage{geometry}
%\usepackage{titlesec}
\usepackage{fancyhdr}
\usepackage[pdfpagelabels=true]{hyperref}
\usepackage{xcolor}
\usepackage{listings}

\definecolor{mGreen}{rgb}{0,0.6,0}
\definecolor{mGray}{rgb}{0.5,0.5,0.5}
\definecolor{airforceblue}{rgb}{0.36, 0.54, 0.66}
\definecolor{amber}{rgb}{1.0, 0.49, 0.0}
\definecolor{coolgrey}{rgb}{0.55, 0.57, 0.67}
\definecolor{davy}{rgb}{0.33, 0.33, 0.33}
\definecolor{backgroundColour}{rgb}{0.95,0.95,0.95}

\lstdefinestyle{CStyle}{
    backgroundcolor=\color{backgroundColour},   
    commentstyle=\color{davy},
    keywordstyle=\color{amber},
    numberstyle=\tiny\color{mGray},
    stringstyle=\color{airforceblue},
    basicstyle=\footnotesize,
    breakatwhitespace=false,         
    breaklines=true,                 
    captionpos=b,                    
    keepspaces=true,                 
    numbers=left,                    
    numbersep=5pt,                  
    showspaces=false,                
    showstringspaces=false,
    showtabs=false,                  
    tabsize=2,
    language=C
}

\geometry{a4paper, top=27mm, left=25mm, right=25mm, bottom=35mm, headsep=10mm, footskip=12mm}

\makeatletter
\renewcommand\@biblabel[1]{}
\makeatother
%__________________________________________________________________

\begin{document}

%\titlespacing{\section}{0pt}{12pt plus 4pt minus 2pt}{-6pt plus 2pt minus 2pt}

% Kopf- und Fusszeile
\pagestyle{fancy}
\renewcommand{\sectionmark}[1]{\markright{\arabic{section}.\ #1}}
\renewcommand{\leftmark}{}
\fancyhf{}
\lhead{}
\chead{}
\rhead{\thesection\space\contentsname}
\lfoot{Abgabe 1}
\cfoot{}
\rfoot{\ \linebreak  \thepage}
\renewcommand{\headrulewidth}{0.4pt}
\renewcommand{\footrulewidth}{0.4pt}

%______________________________________________________________________


%______________________________________________________________________

\clearpage

% Arabische Seitenzahlen
\pagenumbering{arabic}
\setcounter{page}{1}

% Geändertes Format für Seitenränder, arabische Seitenzahlen
\fancyhead[LE,RO]{\rightmark}
%\fancyhead[LO,RE]{\leftmark}
\fancyfoot[LE,RO]{\thepage}

%______________________________________________________________________

\section*{Abgabe 1}

\begin{lstlisting}[style=CStyle]
#include <stdlib.h>
#include <stdio.h>
#include <unistd.h>
#include <pthread.h>
#include <errno.h>

void* function(void* arg) {
	//sleeps the given time
	sleep((int) arg);
	int id = pthread_self();
	printf("This is thread %d\n", id);
	//returns its id
	return (void *) id;
}

int main(int argc, char *argv[]) {
	pthread_t thread_one;
	pthread_t thread_two;
	pthread_attr_t attr;
	pthread_attr_t attr1;
	pthread_t returnValue;
	pthread_t returnValue1;
	int err;

	/*
	 * set attributes
	 * PTHREAD_CREATE_JOINABLE so that the process won't be closed until it is joined.
	 * If PTHREAD_CREATE_DETACHED is used, the main program wouldn't be able to join the thread.
	 * With detached as soon as the thread is finished it gets deleted.
	 * With joinable when the thread is finished it gets preserved until it gets joined.
	 */
	pthread_attr_init(&attr);
	pthread_attr_init(&attr1);
	pthread_attr_setdetachstate(&attr, PTHREAD_CREATE_JOINABLE);
	pthread_attr_setdetachstate(&attr1, PTHREAD_CREATE_JOINABLE);

	//create first thread that sleeps 2 seconds
	err = pthread_create(&thread_one, &attr, &function, (void *) 2);

	if (err != 0) {
		printf("pthread_create: %d ", strerror(err));
	}

	//create second thread that sleeps 4 seconds
	err = pthread_create(&thread_two, &attr1, &function, (void *) 4);
	if (err != 0) {
		printf("pthread_create: %d  ", strerror(err));
	}

	printf("Joining Thread One\n");
	//join first thread and get returned thread id
	err = pthread_join(thread_one, (void **) &returnValue);
	if (err != 0) {
		printf("pthread_join: %d ", strerror(err));
	}
	printf("returnValue %d\n", returnValue);

	printf("Joining Thread Two\n");
	//join second thread and get returned thread id
	err = pthread_join(thread_two, (void **) &returnValue1);
	if (err != 0) {
		printf("pthread_join: %d ", strerror(err));
	}
	printf("returnValue %d\n", returnValue1);

	//compare returned thread id with the one received while creating
	printf("Difference for first thread %d\n", returnValue - thread_one);
	printf("Difference for second thread %d\n", returnValue1 - thread_two);

	return EXIT_SUCCESS;
}


\end{lstlisting}

%______________________________________________________________________

\end{document}












