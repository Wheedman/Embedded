\documentclass[12pt,a4paper,bibliography=totocnumbered]{scrartcl}
\usepackage[ngerman]{babel}

\usepackage{graphicx}
\usepackage{geometry}
%\usepackage{titlesec}
\usepackage{fancyhdr}
\usepackage[pdfpagelabels=true]{hyperref}
\usepackage{xcolor}
\usepackage{listings}
\usepackage[T1]{fontenc}

\definecolor{mGreen}{rgb}{0,0.6,0}
\definecolor{mGray}{rgb}{0.5,0.5,0.5}
\definecolor{airforceblue}{rgb}{0.36, 0.54, 0.66}
\definecolor{amber}{rgb}{1.0, 0.49, 0.0}
\definecolor{coolgrey}{rgb}{0.55, 0.57, 0.67}
\definecolor{davy}{rgb}{0.33, 0.33, 0.33}
\definecolor{backgroundColour}{rgb}{0.95,0.95,0.95}

\lstdefinestyle{CStyle}{
    backgroundcolor=\color{backgroundColour},   
    commentstyle=\color{davy},
    keywordstyle=\color{amber},
    numberstyle=\tiny\color{mGray},
    stringstyle=\color{airforceblue},
    basicstyle=\footnotesize,
    breakatwhitespace=false,         
    breaklines=true,                 
    captionpos=b,                    
    keepspaces=true,                 
    numbers=left,                    
    numbersep=5pt,                  
    showspaces=false,                
    showstringspaces=false,
    showtabs=false,                  
    tabsize=2,
    language=C
}

\geometry{a4paper, top=27mm, left=25mm, right=25mm, bottom=35mm, headsep=10mm, footskip=12mm}

\makeatletter
\renewcommand\@biblabel[1]{}
\makeatother
%__________________________________________________________________

\begin{document}

%\titlespacing{\section}{0pt}{12pt plus 4pt minus 2pt}{-6pt plus 2pt minus 2pt}

% Kopf- und Fusszeile
\pagestyle{fancy}
\renewcommand{\sectionmark}[1]{\markright{\arabic{section}.\ #1}}
\renewcommand{\leftmark}{}
\fancyhf{}
\lhead{}
\chead{}
\rhead{\thesection\space\contentsname}
\lfoot{Abgabe 5}
\cfoot{}
\rfoot{\ \linebreak  \thepage}
\renewcommand{\headrulewidth}{0.4pt}
\renewcommand{\footrulewidth}{0.4pt}

%______________________________________________________________________


%______________________________________________________________________

\clearpage

% Arabische Seitenzahlen
\pagenumbering{arabic}
\setcounter{page}{1}

% Geändertes Format für Seitenränder, arabische Seitenzahlen
\fancyhead[LE,RO]{\rightmark}
%\fancyhead[LO,RE]{\leftmark}
\fancyfoot[LE,RO]{\thepage}

%______________________________________________________________________

\section*{Abgabe 5}

\begin{lstlisting}[style=CStyle]
#include <stdlib.h>
#include <stdio.h>
#include <sys/types.h>
#include <sys/stat.h>
#include <fcntl.h>
#include <errno.h>
#include <string.h>
#include <unistd.h>

const char * path = "/dev/leds";

/**
* Ausgabe eines Bytes in binaer Form.
*/
void printBinary(char * name, char toBinary) {
	int i;
	printf("%s :", name);
	for (i = 0; i < 8; i++) {
		printf("%d", !!((toBinary << i) & 0x80));
	}
	printf("\n");
}

/**
* Beschreiben des LED Pin von BBC mit Binaercodierung.
*/
void writeLEDs(char buffer) {
	int fileDescritpion = open(path, O_WRONLY, S_IWUSR);

	if (fileDescritpion == -1) {
		printf("open: %s ", strerror(errno));
	}
	int writeSize = write(fileDescritpion, &buffer, sizeof(buffer));

	if (writeSize == -1) {
		printf("write: %s ", strerror(errno));
	}
	close(fileDescritpion);
}

/**
* Warten auf Usereingabe mit Dialog zum Setzten der ersten beiden LED's.
* OR und XOR Funktion fuer andere beiden LED's basierend auf Werten von LED 1 und 2.
* Zusammensetzen der LED Binaercodierung in einem Byte mittels Binaerverschiebung.
* Schreibaufruf.
*/
int main(int argc, char *argv[]) {

	int input;
	while (1) {

		printf("Enter led1: ");
		input = getchar();
		while (getchar() != '\n') {
			;
		}

		char led1 = (char) (input - '0');
		printBinary("led1", led1);

		printf("Enter led2: ");
		input = getchar();
		while (getchar() != '\n') {
			;
		}
		char led2 = (char) (input - '0');
		printBinary("led2", led2);

		char led3 = led1 | led2;
		printBinary("led3", led3);

		char led4 = led1 ^ led2;
		printBinary("led4", led4);

#if 0
		led4 = led4 << 1;
		led4 = led4 | led3;
		led4 = led4 << 1;
		led4 = led4 | led2;
		led4 = led4 << 1;
		led4 = led4 | led1;
#else
		led4 = led4 << 3 | led3 << 2 | led2 << 1 | led1;
#endif
		writeLEDs(led4);
		printf("\n");
	}

	return EXIT_SUCCESS;
}
\end{lstlisting}
%______________________________________________________________________

\end{document}












